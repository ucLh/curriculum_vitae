%%%%%%%%%%%%%%%%%
% This is a sample CV template created using altacv.cls
% (v1.1.4, 27 July 2018) written by LianTze Lim (liantze@gmail.com). Now compiles with pdfLaTeX, XeLaTeX and LuaLaTeX.
% 
%% It may be distributed and/or modified under the
%% conditions of the LaTeX Project Public License, either version 1.3
%% of this license or (at your option) any later version.
%% The latest version of this license is in
%%    http://www.latex-project.org/lppl.txt
%% and version 1.3 or later is part of all distributions of LaTeX
%% version 2003/12/01 or later.
%%%%%%%%%%%%%%%%

%% If you need to pass whatever options to xcolor
\PassOptionsToPackage{dvipsnames}{xcolor}

%% If you are using \orcid or academicons
%% icons, make sure you have the academicons 
%% option here, and compile with XeLaTeX
%% or LuaLaTeX.
% \documentclass[10pt,a4paper,academicons]{altacv}

%% Use the "normalphoto" option if you want a normal photo instead of cropped to a circle
% \documentclass[10pt,a4paper,normalphoto]{altacv}

\documentclass[10pt,a4paper]{altacv}
%% AltaCV uses the fontawesome and academicon fonts
%% and packages. 
%% See texdoc.net/pkg/fontawecome and http://texdoc.net/pkg/academicons for full list of symbols.
%% 
%% Compile with LuaLaTeX for best results. If you
%% want to use XeLaTeX, you may need to install
%% Academicons.ttf in your operating system's font 
%% folder.


% Change the page layout if you need to
\geometry{left=1cm,right=9cm,marginparwidth=6.8cm,marginparsep=1.2cm,top=1.25cm,bottom=1.25cm,footskip=2\baselineskip}

% Change the font if you want to.

% If using pdflatex:
% \usepackage[T1]{fontenc}
% \usepackage[utf8]{inputenc}
% \usepackage[default]{lato}

% If using xelatex or lualatex:
\setmainfont{Lato}

% Change the colours if you want to
\definecolor{Navy}{HTML}{000080}
\definecolor{SlateGrey}{HTML}{2E2E2E}
\definecolor{LightGrey}{HTML}{666666}
\colorlet{heading}{Navy}
\colorlet{accent}{Navy}
\colorlet{emphasis}{SlateGrey}
\colorlet{body}{SlateGrey}


% Change the bullets for itemize and rating marker
% for \cvskill if you want to
\renewcommand{\itemmarker}{{\small\textbullet}}
\renewcommand{\ratingmarker}{\faCircle}
%% sample.bib contains your publications
\addbibresource{sample.bib}

\usepackage{hyperref}
\newcommand{\MYhref}[3][Navy]{\href{#2}{\color{#1}{#3}}}%

\begin{document}

\name{Vladimir Luchinskiy}
\tagline{Computer Vision Engineer}
\personalinfo{%
  % Not all of these are required!
  % You can add your own with \printinfo{symbol}{detail}
  \location{Moscow, Russia}
  \email{vladimir.luchinskiy@yandex.ru}
%   \phone{+91 9549497979}
%  \homepage{www.homepage.com}
%  \kaggle{@twitterhandle}
  \github{\href{https://github.com/ucLh}{github.com/ucLh}}
  %% You MUST add the academicons option to \documentclass, then compile with LuaLaTeX or XeLaTeX, if you want to use \orcid or other academicons commands.
%   \orcid{orcid.org/0000-0000-0000-0000}
}

%% Make the header extend all the way to the right, if you want. 
\begin{fullwidth}
\makecvheader
I am a Master's Degree student and a Computer Vision Engineer with 3 years of experience looking for Middle Computer Vision Engineer position in a field of Deep Learning.
\end{fullwidth}

%% Depending on your tastes, you may want to make fonts of itemize environments slightly smaller
% \AtBeginEnvironment{itemize}{\small}


%% Provide the file name containing the sidebar contents as an optional parameter to \cvsection.
%% You can always just use \marginpar{...} if you do
%% not need to align the top of the contents to any
%% \cvsection title in the "main" bar.
\cvsection[page1sidebar]{Work Experience}

\cvevent{Computer Vision Engineer}{\href{https://www.integrant.ru/}{IntegraNT LLC}}{May 2021 -- Present}{Moscow, Russia}

\begin{itemize}
% https://www.mos.ru/news/item/97751073/
\item Working on perception tasks for self-driving \MYhref{https://www.youtube.com/watch?v=nBZdgo9uvz4}{sweeper robot}.
\begin{itemize}
    % \item Trained various versions of YOLO neural network to detect garbage in the city scenes, trained lightweight neural networks for semantic segmentation of city scenes. 
    % \item Successfully performed post-train quantization of UNet to int8 for TensorRT.
    % \item Wrapped TensorRT inference of detection and segmentation networks as ROS (Robotic Operating System) nodes.
    \item Trained 2D one-stage object detection networks (YOLOX, EfficientDet), lightweight 2D (DDRNet, UNet) and 3D (SalsaNext) semantic segmentation networks, wrote optimized inference for them via both Python and C++ TensorRT API. Wrapped inference as ROS (Robotic Operating System) and ROS2 nodes.
    \item Currently, 2 of my networks are deployed on our sweeper robot prototype. The networks were targeted for Jetson TX2 and Jetson Xavier embedded systems. 
    \item Converted convolutional neural networks from Pytorch to TensorRT using ONNX parsers and directly using TensorRT Python API. Successfully performed post-train quantization of UNet segmentation network to int8 for TensorRT.
    \item Created simple monitoring system that evaluates lightweight object detection networks (like YOLOX-tiny) on freshly collected data using automatically generated annotations from pretrained heavyweight detection networks (like EfficientDet-d7). The system was implemented on Python via DVC and MLFlow frameworks.  
    
    % \item Trained and wrote TensorRT inference for 3D semantic segmentation network SalsaNext. 
\end{itemize}

\end{itemize}

\divider

\cvevent{Computer Vision Engineer}{Computer Vision Systems LLC}{Nov 2018 -- Aug 2021}{Saint-Petersburg, Russia}

\begin{itemize}
\item Worked on city objects recognition task in \MYhref{https://www.augmented.city/}{ Augmented.City} project. 
    \begin{itemize}
        \item Trained FaceNet with center loss to perform image retrieval for corresponding task. The network has won a city objects recognition \MYhref{https://fpi.gov.ru/tenders/799/}{competition} held by \MYhref{https://fpi.gov.ru/about/}{{FPI}}.
        \item{Used generative adversarial network (CycleGAN) for the expansion of training data for the task of city objects recognition in different weather conditions. Specifically, the expansion was done by converting summer pictures to their winter counterpart via GAN. This approach improved accuracy on winter data from 31\% to 42\%.}
        % \item {Wrote pipelines using TensorFlow 1.x and tf.data API.}
    \end{itemize}
\item{Trained two-headed Unet-EfficientNet for searching defects on steel pipes using Pytorch. Proved the concept and helped my company to earn a contract with \MYhref{https://www.tmk-group.com/}{{TMK}.}}
\item{Converted Pytorch segmentation and detection models to TensorRT for deploying on NVIDIA Jetson. Created a small C++ library for inferencing these models via TensorRT.}
\item{Speeded up pretrained TensorFlow Lite networks on ARM edge devices with ArmNN library.}
\end{itemize}


% \cvsection{Education}

% \cvevent{Master's degree}{Saint-Petersburg State Univercity, Mathematics and Mechanics faculty, Computer Science department}{2020 -- Present}{Saint-Petersburg, Russia}

% \divider

% \cvevent{Bachelor's degree}{Saint-Petersburg State Univercity, Mathematics and Mechanics faculty, System Programming department}{2016 -- 2020}{Saint-Petersburg, Russia}

% \divider



\medskip


\end{document}
